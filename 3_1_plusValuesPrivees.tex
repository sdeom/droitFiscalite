Chapitre I - Plus-values immobilières privées

Section I- Cession d'immeubles ou de droits relatifs à un immeuble 

		I- Champ d'application

			A- Personnes imposables
				
1- Personnes physiques 
				2- Sociétés de personnes 
			
B- Biens imposables
				
1- Principe : 
					a- Immeubles
					b- Immeuble construit par le cédant
					c- Terrain loti
					d- Droits relatifs à des immeubles

				2- Exceptions 
		
			C- Opérations imposables
		
				1-Ventes et opérations assimilées
				2-  Partages
				3- Licitations
	Cas pratique n°28 : 	Plus-value résultant de la cession d’un immeuble indivis après partage

		II- Exonérations 

			A- Titulaires de pensions de vieillesse ou d'une carte d'invalidité
	
			B- Cession de la résidence principale
		
				1- Conditions de l'exonération
				2- Cas de l’immeuble affecté pour partie à usage professionnel
				3- Cas de la cession d’une ancienne résidence principale 
					a- Principe
					b- Exception
				4- Cas de l’immeuble occupé jusqu'à sa mise en vente
				5-Cas de l’immeuble occupé par le futur acquéreur
		
			C- Logement autre que la résidence principale
		
			D- Expropriations
		
			E- Opérations de remembrement ou assimilées
		
			F- Exonération tenant au montant des cessions
		
				1- Principe 
				2- Exceptions 
a- Cession d'un bien dont le droit de propriété est démembré
b- Cession d'un bien détenu en indivision

			G- Exonération tenant à la durée de possession

		III- Détermination de la plus-value imposable

			A- Calcul de la plus-value brute
		
				1- Prix de cession
					a- Majoration du prix de cession 
					b- Minoration du prix de cession
				2- Prix d'acquisition
					a- Définition du prix d'acquisition
					b- Majoration du prix d'acquisition
		
			B- Calcul de la plus-value imposable

1-Abattement pour durée de détention
2-Moins-values
3- Cas de la vente d'un immeuble construit par le cédant (profits de construction occasionnels)

		IV- Modalités d'imposition 

			A- Fait générateur 
			
B- Calcul de l'impôt
				1-Impôt sur le revenu
				2-Prélèvements sociaux
				3- Taxe sur les plus-values immobilières élevées

		V- Déclaration de plus-value immobilière

			A- Principe
			B- Exceptions

	Cas pratique n°29
	
Section II- Plus-values immobilières des sociétés relevant des articles 8 à 8 ter du CGI


I- Apport d'un immeuble à une société


II- Vente d'un immeuble par la société


A- Champ d'application

B- Exonérations


III- Retrait d'un associé


IV- Dissolution de la société


A- Opérations imposables

B- Calcul de la plus-value 


V- Transformation et changement de régime fiscal



Section III- Plus-values sur cession de titres de sociétés à prépondérance immobilière


I- Champ d'application


A- Personnes concernées

B- Titres de sociétés concernés

1- Titres de sociétés ou groupements relevant des articles 8 à 8 ter du CGI

2- Exclusion des Titres des sociétés soumises à l'impôt sur les sociétés

C- Exonérations

D- Opérations imposables


II - Détermination de la plus-value imposable


A- Prix de cession

B- Prix d'acquisition

C- Délais de possession


Cas pratique n°30


Cas pratique n° 31 : 	Détermination de Plus-values en cas de titres de même nature acquis à des dates et pour des prix différents


D- Moins-values


III- Calcul de l'impôt



