% !TeX root = droitFiscalité.tex

\chapter{Réforme de la fiscalité de l’urbanisme et des territoires}

	C'est une réforme qui résulte de la 4eme loi de finance du 30/12/2010. en generla il y a 2 loi de finance respectificative\footnote{En 2010 il a fallu faire face à la crise de dette de la dette souveraine.}. Il s'agit au départ de suprimer toute les taxes d'urbanisme pour la remplacer par la taxe d'aménagement\footnote{Pareil qu'en 1970 qui avait créé la TLE}.

\section{La taxe d’aménagement}

	\subsection{Création de la taxe d’aménagement}

		Remaques préliminiares :
		\begin{enumerate}
			\item La \TA est institué de plein droit dans les communes disposant d'un PLU et facultative dans les autres communes.
			\item La \TA peut être transférée à un EPCI.
			\item La \TA a trois niveaux : communal, dpt, region (en idf)
		\end{enumerate}

	\subsection{Champ d’application}

		<< La \TA est établie sur la construction, la reconstruction l'agrandissement des bâtiment et l'amngment de toute nature necesiisatnt une autorisation d'urbanisme >>.

		Il y a trois champs d'exonération :
		\begin{enumerate}
			\item exo totales
			\item exo de la part communale ou intercommunale
			\item xonérations facultatives
		\end{enumerate}

			\paragraph{Exonérations totales de la taxe}

				même que pour redevance arché :
	      \begin{enumerate}
	        \item construction destiné à un service public ou d'utilité publique\footnote{établissement de santé ESPIC fevrier 2020} ;
	        \item construction d'habitation ou d'hébargement qui béncie d'un PLAI
	        \item constructions agricoles
	        \item les aménagements prévus par un PPR, tant naturel que technologique
	        \item les reconstructions à l'identiques
	        \item construction dont la surface est inférieure à moins de 5 m2 (pas dans la liste du code mais pas d'autorisation a demander)
	      \end{enumerate}

			\paragraph{Exonérations de la part communale ou intercommunale}

				On peut exonérer de la part communale ou intercommunale dans certains périmétres (ZAC et PUP et QPPP)

				Dans ces cas exonération car vous allez supporter à la palce de la commune

			\paragraph{Exonérations facultatives}

			Dépend de chaque niveau : commune, departement et région.

			\begin{enumerate}
				\item tout ou  partie, des logements sociaux qui bénéficient des tx réduits de TVA
				\item partiellement, des constructions à usage de residence principale qui ne beneficieraient pas de plein droit l'abbatement de 50\% et qui serait financé par un tx 0\footnote{vise les maisons }
				\item tout ou  partie,les constructions industrielle
				\item tout ou  partie,les commerce de détail inf a 400m2
				\item tout ou  partie,les trvx réalisé ur des immaubles classé monuments historiuqe
				\item totale, les abris de jardins
			\end{enumerate}

	\subsection{Base d’imposition}

		L331-10 et suivants

		<< somme des surfaces de plancher clauses et couvertes d'une hauteur supérieur a 1,8m calculée à partir du nu intérieur des façades, deduction faite des vides et des trémies. >>

		Abbatement de \pourcent{50} concernant
		\begin{itemize}
			\item les HLM
			\item residence principale pour les 100 1er mé si situé les en zone U ou situé dans un lotissement
			\item activité économique
		\end{itemize}

		Cas pratique n°4

	\subsection{Taux d’imposition}

		A l'assiette on applique un tx

		3 niveaux donc 3 Taux

		\paragraph{Taux communal ou inter} Il doit varier entre 1 et 5\%

			\textbf{Attention} : les communes peuvent aller au-delà de ce tx, jusqu'à 20\% si les constructions ont lieu dans des secteurs donnant lieu à des travaux substantiels de [Trouver la source]

		\paragraph{Taux départemental} Il ne peut exceder 2,5\%

		\paragraph{Taux régionnal} En Ile-de-France fixé à 1\%.

	\subsection{Etablissement de la taxe}

		La taxe est du par le beneficiare de l'autorisation de construire ou d'aménager.

		C'est les services de l'Etat qui etabliront la taxe et qui la liquideront

	\subsection{Recouvrement de la taxe}

		< 1500 12 mois après la délivrance

		> 1500 appelé en 2 fois : 12 et 24 mois

		Payable dans les 3 semaines sinon majoration

		+ 3\% pour frais d'assiette et de recouvrement \footnote{article}

		Cas pratique n°5
		Cas pratique n°6
		Cas pratique n°7
		Cas pratique n°8
		Cas pratique n°9
		Cas pratique n°10

		Cas pratique n°5

		Construction d'une maison individuelle en zone U de 185 m2 (hors région Ile-de-France) par permis de construire délivré le 1er juillet de l’année dernière.
		Déterminez le montant de la taxe d’aménagement dû, sachant que les collectivités territoriales ont voté le taux maximum d’imposition les concernant.
		Qui paiera ce montant et quand ?

		Assiette
		• Constructions de locaux à usage de résidences principales et leurs annexes, situées en zone U des POS ou PLU -> 100 * 376,5 = 37650
		85 * 753 = 64005
		= 101655

		tx -> 7,5

		=> 7624

		Frais de recouvrement

		Qui : le beneiciaire du permis
		Quand : + de 1500 => appel en 2 fois 12 mois et 24 mois

Cas Pratique n°6 :

Permis de construire délivré le 1er juillet de l’année dernière autorisant la réalisation d'un ensemble de 30 constructions individuelles représentant une surface totale habitable de 6 300 m2, dont 10 de 90 m2, 10 de 190 m2 et 10 de 350 m2 (hors région Ile-de-France).
Déterminez le montant de la taxe d’aménagement dû, sachant que les collectivités territoriales ont voté le taux maximum d’imposition les concernant.
Qui paiera ce montant et quand ?


		Assiette :
		Constructions de locaux à usage
	de résidences principales et leurs annexes, situées en zone U des POS ou PLU
		par constrcution
		90m2
			90 * 376,5 = 33885

			x 10 = 338850

		190mé
			100 * 376,5 = 37650
			90 * 753 = 67 770
				= 105420

				x 10 = 1054200

		350m2
		100 * 376,5 = 37650
		150 * 753 = 112950
			= 150600

			x 10 = 1506000

			=> 2 899 050 -> non 3152020

		Taux :
			Non 20\% + 2,5 \%
			=> 22,5

			=> 821 711

			En 2 fois + 3%

Cas pratique n°7 :

Construction d'un immeuble collectif de 30 logements à titre de résidence principale de 100 m2 chacun en Ile-de-France par permis de construire délivré le 1er juillet de l’année dernière.
Déterminez le montant de la taxe d’aménagement dû, sachant que les collectivités territoriales ont voté le taux maximum d’imposition les concernant.
Qui paiera ce montant et quand ?

	Pas de necessité trvx donc pas 20\%

	=> 108885

Cas pratique n°8:

Monsieur X particulier, envisage la création d’un plancher supplémentaire de 100m2 à l’intérieur d’une grange réhabilité en 2000 à usage de résidence secondaire, en IDF.
Il souhaite également transformer son garage de 20 m2 en chambre d’amis.
Déterminez le montant de la taxe d’aménagement dû, sachant que les collectivités territoriales ont voté le taux maximum d’imposition les concernant.
Qui paiera ce montant et quand ?

	Taux : pas de becesiité de trvx Idf => " 3 nvx " tx = 8,5

	Assiette : redience secondaire => 854 * 100 = 85400 Il n'y a pas les 20m2 car non soumis à autorisation

	=> 7259

	261


Cas pratique n°9 :

Monsieur Y a construit le 15 juin de l’année dernière un abri de jardin de 8 m2 dans sa résidence principale en IDF.
Déterminez le montant de la taxe d’aménagement dû, sachant que les collectivités territoriales ont voté le taux maximum d’imposition les concernant.
Qui paiera ce montant et quand ?

	Tout dépend de la surface déjà consommé pour la maison. 598 euros...


Cas pratique n°10 :

La société Y exploitante de supermarchés souhaite réaliser une extension de l’un des parkings jouxtant l’un de ses magasins en IDF. Ce parking non clos, non couvert de 1.500 M2, sera destiné à accueillir 80 emplacements.
Déterminez le montant de la taxe d’aménagement dû, sachant que les collectivités territoriales ont voté le taux maximum d’imposition les concernant.
Qui paiera ce montant et quand ?


\section{Versement pour sous-densité (VSD)}

	Il s'agit d'un impot local.
	Date de 2012. Annonce la fin de la zone pavillonaire. En effet, beaucoups de \colloc ont recours aux taux supérieur de la \ta.

	Institué par arreté du maire sur décision du \CM.

	Paiement tout pareil que TA

	\begin{enumerate}
		\item réservé aux zones U ou AU des plu ou pos
		\item si la commune choisi de l'instaureer, le versement pour dépassement du plafond légal de densité (l'ancien) disparait automatiquement
		\item transférable à l'EPCI
		\item le versement est égal
			<< {\itshape au produit de la moitié de la valeur du terrain par le rapport entre la surface manquante pour que la consrtruction atteigne le seuil minimal de densité et la surface de la construction résultant de ce seuil} >>

			\[[ Vsd = 50\% x Vt x SM/SCAR \]]
		\item ce prélevement est facultatif , mais les communes sont obligé si elles ont dépassé le tx de 5\% de taxe d'aménagement (cf. \ta et faculté de la porter jusqu'à 20\%)

		\item il est plafonné à \pourcent{25} de la valeur du terrain nu constructible.
	\end{enumerate}

Cas pratique n°11 :


Monsieur et Madame Z envisagent de construire une maison de 100 m2 sur un terrain de 2.000 m2 acquis pour 100.000 €.

SCAR = 4000m2
SM = 3900m2

VT=1000000
\[[ (0,5 x 1000000)\]]
=> 39 400/ x 50 000n= 49 000

1/4 terrain nu = 25000

=> 25 000

Combien paieront-ils au titre du versement pour sous-densité, étant entendu que le PSD voté est de 2 ?
