Chapitre II- L’imposition des revenus fonciers

	Section I – Régime général d’imposition des revenus fonciers 

		I- Champ d'application
			A- Revenus tirés de la location de l’immeuble
			B- Revenus accessoires
			C- Territorialité

		II- Revenu imposable 

			A-  Régime réel d'imposition
				1- Recettes
				2- Dépenses

			B- Régime du Micro-foncier 
				1- Champ d'application
				2- Modalités d'imposition

		III- Imputation des déficits fonciers 
		
			A- Principes
			B- Exceptions 
			C- Imputation des déficits fonciers, Décision de rescrit 22-6-2010 				n°2010/35 

			Cas pratique n°12

	Section II – Régime d’investissement DUFLOT - PINEL

I- Champ d’application

A. Personnes concernées

B. Investissements concernés

1. Acquisition d'un immeuble

a. Acquisitions de logements neufs achevés
b. Acquisitions de logements en l'état futur d'achèvement
c. Acquisition d'un logement en vue de sa réhabilitation
d. Acquisition d'un logement réhabilité
e. Acquisition d'un local que le contribuable transforme en habitation
f. Acquisition d'un logement issu de la transformation d'un local affecté à un autre usage
g. Acquisition d'un logement qui a fait ou qui fait l'objet de travaux concourant à la production ou à la livraison d'un immeuble neuf au sens de la TVA
h. Logement que le contribuable fait construire
i. Acquisition d'un local inachevé

2. Souscription de parts de SCPI

C. Caractéristiques des immeubles

1. Immeuble à usage de logement

2. Situation de l'immeuble 
a. Investissements réalisés en métropole
b. Investissements réalisés outre-mer

3. Performance énergétique globale des logements
a. Logements situés en métropole
b. Investissements réalisés outre-mer


II- Conditions d’application de la réduction d’impôt

A. Engagement de location

1. Engagement du propriétaire du logement
2. Contenu de l'engagement de location
3. Constatation de l'engagement de location

B. Engagement de conservation des parts

Cas pratique n°13
C. Conditions de mise en location
1. Délai de mise en location
2. Durée de location
3. Affectation des logements
4. Qualité du locataire
5. Location à un organisme public ou privé en vue de sa sous-location

D. Plafonds de loyer

1. Investissements réalisés en métropole
a. Plafonds de loyer mensuel par mètre carré
b. Surface à prendre en compte

Cas pratique n°14

2. Investissements réalisés outre-mer

3. Dispositions communes : le coefficient multiplicateur


E. Plafonds de ressources des locataires

1. Investissements réalisés en métropole
2. Investissements réalisés outre-mer
3. Dispositions communes

a. Obligations déclaratives
b. Ressources à prendre en compte
c. Location à un organisme public ou privé en vue de sa sous-location

III- Modalités d’application de la réduction d’impôt

A. Base de la réduction d'impôt

1. Acquisition de logements
a. Détermination de la base de la réduction d'impôt
b. Plafond de la base de la réduction d'impôt
c. Plafond de prix de revient par mètre carré de surface habitable
d. Articulation entre les plafonds 

Cas pratique n°15
Cas pratique n°16
Cas pratique n°17 

2. Souscription de parts de SCPI

a. Plafond de la base de réduction d'impôt
b. Acquisition en indivision

3. Acquisition d'un logement et souscription de parts de SCPI au titre d'une même année

Cas pratique n° 18 

B. Taux de la réduction d'impôt

C. Modalités d'imputation de la réduction d'impôt

D. Limitation du nombre d'investissements


Cas pratique n°19 
Cas pratique n°20
Cas pratique n°21 

E. Règles de non-cumul

F. Articulation avec d'autres dispositifs

Cas pratique n°22 

IV- Obligations

A. Logements acquis directement par le contribuable

B. Logements acquis par l'intermédiaire d'une société non soumise à l'IS
1. Obligations des sociétés
2. Obligations des associés

C. Souscription de parts de SCPI

1. Obligations des sociétés
2. Obligations des associés
