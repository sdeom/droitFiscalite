% !TeX root = droitFiscalité.tex
\chapter{L’imposition des revenus fonciers}

\begin{table}
	\centering
	\begin{tabular}{ll}
		\toprule
			\textbf{Tranche du revenu 2019} & \textbf{Taux d’imposition} \\
			(Quotient familial) & (TMI) \\
		\midrule
			Jusqu'à \nombre{10064} euros &	\pourcent{0} \\
			de \nombre{10064} à \nombre{27794} euros &	\pourcent{14} \\
			de \nombre{27794} à \nombre{74517} euros &	\pourcent{30} \\
			de \nombre{74517} à \nombre{157806} euros & 	\pourcent{41} \\
			Supérieur à \nombre{157806} euros &	\pourcent{45} \\
		\bottomrule
	\end{tabular}
	\label{tab:impotRevenusFonciers}
	\caption{L’imposition des revenus fonciers}
\end{table}

\begin{table}
	\centering
	\begin{tabular}{ll}
		\toprule
			CSG & \pourcent{9,2} \\
			CRDS &	\pourcent{0,5} \\
			Prélèvement de solidarité & \pourcent{7,5} \\
		\midrule
			Total : & \pourcent{17,2}
		\bottomrule
	\end{tabular}
	\label{tab:prelevementsSociaux}
	\caption{Prélèvements sociaux}
\end{table}

\section{Régime général d’imposition des revenus fonciers}

	\subsection{Champ d'application}

			\subsubsection{Revenus tirés de la location de l’immeuble}

				On appelle les revenus fonciers les revenus :
				\begin{itemize}
					\item de le location de propriété bâtie (\etc) ;
					\item de la location de prop non bâtie (pré champs, bois carrière, \etc) ;
					\item perçus par des organisme << translucides >> qui ont une personnalité juridiques mais pas de personnalité fiscales (sci, société en commandite, \etc)
				\end{itemize}

			\subsubsection{Revenus accessoires}

				Il faut ajouter les revenus accessoires. Ce sont tous le srevenus assimilés à des revenus fonciers car liés à une location qui porte sur l'immeuble.

				Exemple : droit d'affichage, droit de chasse ou de peche, divers droit d'exploitation. Ce peut etre les redevances tréfoncieres (sur Paris principalement)

			\subsubsection{Territorialité}

				En France, il existe un principe d'universalité fiscale. Vous êtes imposés sur vos revenus fonciers mondiaux.

				Pour les non résidents fiscaux, les revenus fonciers des immeubles situés en France sont imposés en France.

				Pour éviter la double imposition, il existe des conventions fiscales internationales. Ce sont des traités, qui ont donc une valeur supra legislative. Il existe 190 conventiosn signée par la France. L'OCDE propose un modèle por harmoniser, et c'est le cas . Lieu d'installation de l'immeuble.

	\subsection{Revenu imposable}

		Etabli à l'\article{28}{\cgi}.

		\subsubsection{Régime réel d'imposition}

			La mécanisme est simple : recette moins dépense.

			\paragraph{Recettes} La notion de recette est codifié à l'\articleDu{29}{\cgi} : << recette de toute nature perçue par le propriétaire >>.

				Ce peut être :
				\begin{itemize}
					\item les loyers, les fermages,
					\item les revenus accessoires,
					\item les dépenses qui incombent au bailleur mais mise à la charge du locataire (),
					\item certaine recettes exceptionnelles,
					\item la valeur des avantages en nature stipulés au bail.
				\end{itemize}

				\medbreak La question du loyer inférieur aux normes habituelles pose problème. Que faire si le propriétaire ...

				Le principe est qu'un loyer inférieur au marché ne peut pas justifier une rectification. Mais cela ne s'applique pas si les circonstances jettent un doute sérieux sur la réalité des conventions ou sur l'authenticité du loyer.

				Il y a des doutes sur la réalité des conventions lorsque le locateur est un proche du bailleur. Les juges considèrent que c'est au contribuable de prouver à l'administration de la réalité et du sérieux des conventions. Il y a donc un inversement de la charge de la preuve\footnote{\jurisCE[48486 ? a verifier]{23/6/1986}}.

				\begin{conseil}
					\begin{enumerate}
						\item Il vaut mieux loger gratuitement qu'à petit loyer.
						\item En cas de contrôle, on peut toujours faire valoir la valeur locative cadastrale.
					\end{enumerate}
				\end{conseil}

				\medbreak La question des aménagements qui reviennent au propriétaire aux termes du bail se pose également.

				\begin{exemple}
					La station service construire sur le terrain d'un agriculteur.

					A perdu en première instance et gagné devant le \CE.
				\end{exemple}

				\medbreak Les dépenses qui incombent au bailleur mais mise à la charge du locataire. Certe ceu sont des recettes, mais ceux sont aussi des dépenses pour le bailleur. Il faut dire à l'administartion que la recette est aussi une charge.

				\medbreak Les recettes exceptionnelles. Le pas de porte par exemple.

				\begin{conseil}
						Cas du bâti construit aux termes d'un bail construction devenant propriété du bailleur à l'extinction du bail ?
						Il ne faut

						La technique est de dire qu'à la fin l'immeuble ne vaut plus rien
				\end{conseil}

			\paragraph{Dépenses}

				Prévu \articleDu{31 par. 1}{\cgi}
				\begin{quote}
					Les charges de la propriété déductibles pour la détermination du revenu net comprennent :

	1° Pour les propriétés urbaines :

	a) Les dépenses de réparation et d'entretien effectivement supportées par le propriétaire ;

	a bis) les primes d'assurance ;

	a ter) Le montant des dépenses supportées pour le compte du locataire par le propriétaire dont celui-ci n'a pu obtenir le remboursement, au 31 décembre de l'année du départ du locataire ;

	a quater) Les provisions pour dépenses, comprises ou non dans le budget prévisionnel de la copropriété, prévues à l'article 14-1 et au I de l'article 14-2 de la loi n° 65-557 du 10 juillet 1965 fixant le statut de la copropriété des immeubles bâtis, supportées par le propriétaire, diminuées du montant des provisions déduites l'année précédente qui correspond à des charges non déductibles ;

	b) Les dépenses d'amélioration afférentes aux locaux d'habitation, à l'exclusion des frais correspondant à des travaux de construction, de reconstruction ou d'agrandissement, ainsi que des dépenses au titre desquelles le propriétaire bénéficie du crédit d'impôt sur le revenu prévu à l'article 200 quater ou de celui prévu à l'article 200 quater A ;

	b bis) Les dépenses d'amélioration afférentes aux locaux professionnels et commerciaux destinées à protéger ces locaux des effets de l'amiante ou à faciliter l'accueil des handicapés, à l'exclusion des frais correspondant à des travaux de construction, de reconstruction ou d'agrandissement ;


	c) Les impositions, autres que celles incombant normalement à l'occupant, perçues, à raison desdites propriétés, au profit des collectivités territoriales, de certains établissements publics ou d'organismes divers, à l'exception de la taxe annuelle sur les locaux à usage de bureaux, les locaux commerciaux et les locaux de stockage perçue dans la région d'Ile-de-France prévue à l'article 231 ter ;

	d) Les intérêts de dettes contractées pour la conservation, l'acquisition, la construction, la réparation ou l'amélioration des propriétés, y compris celles dont le contribuable est nu-propriétaire et dont l'usufruit appartient à un organisme d'habitations à loyer modéré mentionné à l'article L. 411-2 du code de la construction et de l'habitation, à une société d'économie mixte ou à un organisme disposant de l'agrément prévu à l'article L. 365-1 du même code ;

	e) Les frais de gestion, fixés à 20 € par local, majorés, lorsque ces dépenses sont effectivement supportées par le propriétaire, des frais de rémunération des gardes et concierges, des frais de procédure et des frais de rémunération, honoraire et commission versés à un tiers pour la gestion des immeubles ;

	e bis) Les dépenses supportées par un fonds de placement immobilier mentionné à l'article 239 nonies au titre des frais de fonctionnement et de gestion à proportion des actifs mentionnés au a du 1° du II de l'article L. 214-81 du code monétaire et financier détenus directement ou indirectement par le fonds, à l'exclusion des frais de gestion variables perçus par la société de gestion mentionnée à l'article L. 214-61 du même code en fonction des performances réalisées.

	Les frais de gestion, de souscription et de transaction supportés directement par les porteurs de parts d'un fonds de placement immobilier mentionné à l'article 239 nonies ne sont pas compris dans les charges de la propriété admises en déduction ;

	f) pour les logements situés en France, acquis neufs ou en l'état futur d'achèvement entre le 1er janvier 1996 et le 31 décembre 1998 et à la demande du contribuable, une déduction au titre de l'amortissement égale à 10 % du prix d'acquisition du logement pour les quatre premières années et à 2 % de ce prix pour les vingt années suivantes. La période d'amortissement a pour point de départ le premier jour du mois de l'achèvement de l'immeuble ou de son acquisition si elle est postérieure.
				\end{quote}


			Elle n'est pas limitatrice (article 13 )

			\begin{itemize}
				\item Sur les dépenses d'améliorations ... (interruption)
			\end{itemize}

			\subsubsection{Régime du Micro-foncier} ARRET DU COURS ICI

				1- Champ d'application
				\paragraph{Champ d'application}

				2- Modalités d'imposition
				\paragraph{Modalités d'imposition}


	\subsection{Imputation des déficits fonciers}

			A- Principes
		\subsubsection{Principes}

		B- Exceptions
		\subsubsection{Exceptions}

		C- Imputation des déficits fonciers, Décision de rescrit 22-6-2010 				n°2010/35
		\subsubsection{Imputation des déficits fonciers, Décision de rescrit 22-6-2010 				n°2010/35}


		Cas pratique n°12 : Calcul d’impôt sur le revenu et de prélèvements sociaux.

		Une personne célibataire a pour 2017 un revenu global de 115.000 €, hors un revenu foncier net de 53.000 €.

		Elle dispose d’un déficit foncier de 2012 non utilisé de 3.000 €

		Combien cette personne paiera-t-elle d’impôt sur le revenu et de prélèvements sociaux au titre de ses revenus 2017 ?


\section{Régime d’investissement \nom{Duflot} - \nom{Pinel}}

	\subsection{Champ d’application}

A. Personnes concernées
		\subsubsection{Personnes concernées}


B. Investissements concernés
		\subsubsection{Investissements concernés}


1. Acquisition d'un immeuble
			\paragraph{Acquisition d'un immeuble}


a. Acquisitions de logements neufs achevés
b. Acquisitions de logements en l'état futur d'achèvement
c. Acquisition d'un logement en vue de sa réhabilitation
d. Acquisition d'un logement réhabilité
e. Acquisition d'un local que le contribuable transforme en habitation
f. Acquisition d'un logement issu de la transformation d'un local affecté à un autre usage
g. Acquisition d'un logement qui a fait ou qui fait l'objet de travaux concourant à la production ou à la livraison d'un immeuble neuf au sens de la TVA
h. Logement que le contribuable fait construire
i. Acquisition d'un local inachevé

2. Souscription de parts de SCPI
			\paragraph{Souscription de parts de SCPI}


C. Caractéristiques des immeubles
		\subsubsection{Caractéristiques des immeubles}


1. Immeuble à usage de logement
			\paragraph{Immeuble à usage de logement}


2. Situation de l'immeuble
			\paragraph{Situation de l'immeuble}

a. Investissements réalisés en métropole
b. Investissements réalisés outre-mer

3. Performance énergétique globale des logements
			\paragraph{Performance énergétique globale des logements}

a. Logements situés en métropole
b. Investissements réalisés outre-mer

	\subsection{Conditions d’application de la réduction d’impôt}

A. Engagement de location
		\subsubsection{Engagement de location}


1. Engagement du propriétaire du logement
			\paragraph{Engagement du propriétaire du logement}

2. Contenu de l'engagement de location
			\paragraph{Contenu de l'engagement de location}

3. Constatation de l'engagement de location
			\paragraph{Constatation de l'engagement de location}


B. Engagement de conservation des parts
		\subsubsection{Engagement de conservation des parts}


Cas pratique n°13

Un contribuable réalise une souscription au capital d'une SCPI le 1er juin 2013 avec engagement de conservation des titres pendant 9 ans.

Cette souscription est affectée au financement de cent trente logements.

Les contrats de location prennent effet au 18 novembre 2013 pour le premier de ces logements et au 1er janvier 2014 pour le dernier.

Jusqu’à quelle date le contribuable doit-il conserver ses parts pour bénéficier du dispositif ?

C. Conditions de mise en location
			\subsubsection{Conditions de mise en location}

1. Délai de mise en location
				\paragraph{Délai de mise en location}

2. Durée de location
				\paragraph{Durée de location}

3. Affectation des logements
				\paragraph{Affectation des logements}

4. Qualité du locataire
				\paragraph{Qualité du locataire}

5. Location à un organisme public ou privé en vue de sa sous-location
				\paragraph{Location à un organisme public ou privé en vue de sa sous-location}


D. Plafonds de loyer
			\subsubsection{Plafonds de loyer}


1. Investissements réalisés en métropole
				\paragraph{Investissements réalisés en métropole}

a. Plafonds de loyer mensuel par mètre carré
					\subparagraph{Plafonds de loyer mensuel par mètre carré}

b. Surface à prendre en compte
					\subparagraph{Surface à prendre en compte}


Cas pratique n°14
Une maison individuelle de 90 m2 de surface habitable comporte des combles aménageables d'une surface de 50 m2 et un sous-sol de 90 m2 dont une partie est accessible au stationnement d'un véhicule.

Quelle surface faut-il retenir pour l’appréciation du plafond de loyer ?


2. Investissements réalisés outre-mer
				\paragraph{Investissements réalisés outre-mer}


3. Dispositions communes : le coefficient multiplicateur
				\paragraph{Dispositions communes : le coefficient multiplicateur}



E. Plafonds de ressources des locataires
			\subsubsection{Plafonds de ressources des locataires}


1. Investissements réalisés en métropole
				\paragraph{Investissements réalisés en métropole}

2. Investissements réalisés outre-mer
				\paragraph{Investissements réalisés outre-mer}

3. Dispositions communes
				\paragraph{Dispositions communes}


a. Obligations déclaratives
					\subparagraph{Obligations déclaratives}

b. Ressources à prendre en compte
					\subparagraph{Ressources à prendre en compte}

c. Location à un organisme public ou privé en vue de sa sous-location
					\subparagraph{Location à un organisme public ou privé en vue de sa sous-location}


	\subsection{Modalités d’application de la réduction d’impôt}

III- Modalités d’application de la réduction d’impôt

A. Base de la réduction d'impôt
		\subsubsection{Base de la réduction d'impôt}


1. Acquisition de logements
				\paragraph{Acquisition de logements}

a. Détermination de la base de la réduction d'impôt
					\subparagraph{Détermination de la base de la réduction d'impôt}

b. Plafond de la base de la réduction d'impôt
					\subparagraph{Plafond de la base de la réduction d'impôt}

c. Plafond de prix de revient par mètre carré de surface habitable
					\subparagraph{Plafond de prix de revient par mètre carré de surface habitable}

d. Articulation entre les plafonds
					\subparagraph{Articulation entre les plafonds}


Cas pratique n°15
Un contribuable acquiert un logement pour la somme de 300 000 €.

Le prix de revient par mètre carré de surface habitable est de 7 500 €/m2 et la surface habitable du logement est de 40 m2.

Quel montant doit-on retenir pour le calcul de la réduction d’impôt ?

Cas pratique n°16

Un contribuable acquiert un logement pour la somme de 525 000 €.

Le prix de revient par mètre carré de surface habitable est de 7 500 €/m2 et la surface habitable du logement est de 70 m2.

Quel montant doit-on retenir pour le calcul de la réduction d’impôt ?

Cas pratique n°17 :

Un contribuable acquiert deux logements au titre d'une même année.

Le premier logement est acquis pour la somme de 203 000 €.

Son prix de revient par mètre carré de surface habitable est de 5 800 €/m2 et sa surface habitable est de 35 m2.

Le deuxième logement est acquis pour la somme de 180 000 €. Son prix de revient par mètre carré de surface habitable est de 6 000 €/m2 et sa surface habitable est de 30 m2.

Quel montant doit-on retenir pour le calcul de la réduction d’impôt ?


2. Souscription de parts de SCPI
				\paragraph{Souscription de parts de SCPI}


a. Plafond de la base de réduction d'impôt
					\subparagraph{Plafond de la base de réduction d'impôt}

b. Acquisition en indivision
					\subparagraph{Acquisition en indivision}


3. Acquisition d'un logement et souscription de parts de SCPI au titre d'une même année
				\paragraph{Acquisition d'un logement et souscription de parts de SCPI au titre d'une même année}


				Cas pratique n° 18 :

				Au titre d'une même année d'imposition, un contribuable acquiert un logement et réalise une souscription de parts de SCPI.

				Le logement est acquis pour la somme de 350 000 €.

				Son prix de revient par mètre carré de surface habitable est de 7 500 €/m2 et sa surface habitable est de 50 m2.

				Le montant de la souscription de parts de SCPI s'élève à 50 000 €.

				Quel montant doit-on retenir pour le calcul de la réduction d’impôt ?


B. Taux de la réduction d'impôt
			\subsubsection{Taux de la réduction d'impôt}


C. Modalités d'imputation de la réduction d'impôt
			\subsubsection{Modalités d'imputation de la réduction d'impôt}


D. Limitation du nombre d'investissements
			\subsubsection{Limitation du nombre d'investissements}



			Cas pratique n°19 :

			Un contribuable acquiert en 2017 deux logements en l'état futur d'achèvement situés en métropole pour un prix total de 350 000 €.

			Ce même contribuable acquiert, en 2018, deux autres logements en l'état futur d'achèvement situés outre-mer pour un prix total de 320 000 €.

			L'achèvement de ces quatre logements, qui correspond au fait générateur de la réduction d'impôt, intervient au cours de l'année 2019. Le prix de revient par mètre carré de surface habitable des logements pris en exemple est réputé être inférieur au plafond de 5 500 €

			Quel sera le montant de la réduction d'impôt imputable au maximum ?

Cas pratique n°20

Deux personnes physiques constituent à parts égales une société civile non soumise à l'impôt sur les sociétés.

Cette société acquiert cette année deux logements neufs achevés situés en métropole pour un prix de 240 000 €.

Quel sera le montant de la réduction d'impôt imputable par chaque associé au maximum ?

Cas pratique n°21 :

Deux personnes physiques constituent à parts égales une société civile, non soumise à l'impôt sur les sociétés.

Cette société acquiert cette année trois logements neufs achevés situés en métropole pour un prix total de 390 000 € (150 000 € pour le premier logement, 140 000 € pour le deuxième et 100 000 € pour le dernier).

Quel sera le montant de la réduction d'impôt imputable par chaque associé?


E. Règles de non-cumul
			\subsubsection{Règles de non-cumul}


F. Articulation avec d'autres dispositifs
			\subsubsection{Articulation avec d'autres dispositifs}


			Cas pratique n°22 :

			Un contribuable acquiert cette année un logement en vue de sa réhabilitation pour un montant de 150 000 €.

			Le montant des travaux de réhabilitation s'élève à 50 000 €, qui pour partie ouvrent droit pour le contribuable au crédit d'impôt prévu à l'article 200 quater du CGI pour un montant (en droits) de 1 200 €.

			Quel sera le prix de revient à prendre en compte pour le calcul de la réduction d’impôt ?


	\subsection{Obligations}
IV- Obligations
		\subsubsection{Logements acquis directement par le contribuable}

A. Logements acquis directement par le contribuable

		\subsubsection{Logements acquis par l'intermédiaire d'une société non soumise à l'IS}
B. Logements acquis par l'intermédiaire d'une société non soumise à l'IS

1. Obligations des sociétés
2. Obligations des associés

		\subsubsection{Souscription de parts de SCPI}
C. Souscription de parts de SCPI

			\paragraph{Obligations des sociétés}
1. Obligations des sociétés

			\paragraph{Obligations des associés}

2. Obligations des associés
