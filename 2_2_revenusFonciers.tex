% !TeX root = droitFiscalité.tex
\chapter{L’imposition des revenus fonciers}

\begin{table}
	\centering
	\begin{tabular}{ll}
		\toprule
			\textbf{Tranche du revenu 2019} & \textbf{Taux d’imposition} \\
			(Quotient familial) & (TMI) \\
		\midrule
			Jusqu'à \nombre{10064} euros &	\pourcent{0} \\
			de \nombre{10064} à \nombre{27794} euros &	\pourcent{14} \\
			de \nombre{27794} à \nombre{74517} euros &	\pourcent{30} \\
			de \nombre{74517} à \nombre{157806} euros & 	\pourcent{41} \\
			Supérieur à \nombre{157806} euros &	\pourcent{45} \\
		\bottomrule
	\end{tabular}
	\label{tab:impotRevenusFonciers}
	\caption{L’imposition des revenus fonciers}
\end{table}

\begin{table}
	\centering
	\begin{tabular}{ll}
		\toprule
			CSG & \pourcent{9,2} \\
			CRDS &	\pourcent{0,5} \\
			Prélèvement de solidarité & \pourcent{7,5} \\
		\midrule
			Total : & \pourcent{17,2}
		\bottomrule
	\end{tabular}
	\label{tab:prelevementsSociaux}
	\caption{Prélèvements sociaux}
\end{table}

\section{Régime général d’imposition des revenus fonciers}

	\subsection{Champ d'application}

			\subsubsection{Revenus tirés de la location de l’immeuble}

				On appelle les revenus fonciers les revenus :
				\begin{itemize}
					\item de le location de propriété bâtie (\etc) ;
					\item de la location de prop non bâtie (pré champs, bois carrière, \etc) ;
					\item perçus par des organisme << translucides >> qui ont une personnalité juridiques mais pas de personnalité fiscales (sci, société en commandite, \etc)
				\end{itemize}

			\subsubsection{Revenus accessoires}

				Il faut ajouter les revenus accessoires. Ce sont tous le srevenus assimilés à des revenus fonciers car liés à une location qui porte sur l'immeuble.

				Exemple : droit d'affichage, droit de chasse ou de peche, divers droit d'exploitation. Ce peut etre les redevances tréfoncieres (sur Paris principalement)

			\subsubsection{Territorialité}

				En France, il existe un principe d'universalité fiscale. Vous êtes imposés sur vos revenus fonciers mondiaux.

				Pour les non résidents fiscaux, les revenus fonciers des immeubles situés en France sont imposés en France.

				Pour éviter la double imposition, il existe des conventions fiscales internationales. Ce sont des traités, qui ont donc une valeur supra legislative. Il existe 190 conventiosn signée par la France. L'OCDE propose un modèle por harmoniser, et c'est le cas . Lieu d'installation de l'immeuble.

	\subsection{Revenu imposable}

		Etabli à l'\article{28}{\cgi}.

		\subsubsection{Régime réel d'imposition}

			La mécanisme est simple : recette moins dépense.

			\paragraph{Recettes} La notion de recette est codifié à l'\articleDu{29}{\cgi} : << recette de toute nature perçue par le propriétaire >>.

				Ce peut être :
				\begin{itemize}
					\item les loyers, les fermages,
					\item les revenus accessoires,
					\item les dépenses qui incombent au bailleur mais mise à la charge du locataire (),
					\item certaine recettes exceptionnelles,
					\item la valeur des avantages en nature stipulés au bail.
				\end{itemize}

				\medbreak La question du loyer inférieur aux normes habituelles pose problème. Que faire si le propriétaire ...

				Le principe est qu'un loyer inférieur au marché ne peut pas justifier une rectification. Mais cela ne s'applique pas si les circonstances jettent un doute sérieux sur la réalité des conventions ou sur l'authenticité du loyer.

				Il y a des doutes sur la réalité des conventions lorsque le locateur est un proche du bailleur. Les juges considèrent que c'est au contribuable de prouver à l'administration de la réalité et du sérieux des conventions. Il y a donc un inversement de la charge de la preuve\footnote{\jurisCE[48486 ? a verifier]{23/6/1986}}.

				\begin{conseil}
					\begin{enumerate}
						\item Il vaut mieux loger gratuitement qu'à petit loyer.
						\item En cas de contrôle, on peut toujours faire valoir la valeur locative cadastrale.
					\end{enumerate}
				\end{conseil}

				\medbreak La question des aménagements qui reviennent au propriétaire aux termes du bail se pose également.

				\begin{exemple}
					La station service construire sur le terrain d'un agriculteur.

					A perdu en première instance et gagné devant le \CE.
				\end{exemple}

				\medbreak Les dépenses qui incombent au bailleur mais mise à la charge du locataire. Certe ceu sont des recettes, mais ceux sont aussi des dépenses pour le bailleur. Il faut dire à l'administartion que la recette est aussi une charge.

				\medbreak Les recettes exceptionnelles. Le pas de porte par exemple.

				\begin{conseil}
						Cas du bâti construit aux termes d'un bail construction devenant propriété du bailleur à l'extinction du bail ?
						Il ne faut

						La technique est de dire qu'à la fin l'immeuble ne vaut plus rien
				\end{conseil}

			\paragraph{Dépenses}

				Prévu \articleDu{31 par. 1}{\cgi}
				\begin{quote}
					Les charges de la propriété déductibles pour la détermination du revenu net comprennent :

	1° Pour les propriétés urbaines :

	a) Les dépenses de réparation et d'entretien effectivement supportées par le propriétaire ;

	a bis) les primes d'assurance ;

	a ter) Le montant des dépenses supportées pour le compte du locataire par le propriétaire dont celui-ci n'a pu obtenir le remboursement, au 31 décembre de l'année du départ du locataire ;

	a quater) Les provisions pour dépenses, comprises ou non dans le budget prévisionnel de la copropriété, prévues à l'article 14-1 et au I de l'article 14-2 de la loi n° 65-557 du 10 juillet 1965 fixant le statut de la copropriété des immeubles bâtis, supportées par le propriétaire, diminuées du montant des provisions déduites l'année précédente qui correspond à des charges non déductibles ;

	b) Les dépenses d'amélioration afférentes aux locaux d'habitation, à l'exclusion des frais correspondant à des travaux de construction, de reconstruction ou d'agrandissement, ainsi que des dépenses au titre desquelles le propriétaire bénéficie du crédit d'impôt sur le revenu prévu à l'article 200 quater ou de celui prévu à l'article 200 quater A ;

	b bis) Les dépenses d'amélioration afférentes aux locaux professionnels et commerciaux destinées à protéger ces locaux des effets de l'amiante ou à faciliter l'accueil des handicapés, à l'exclusion des frais correspondant à des travaux de construction, de reconstruction ou d'agrandissement ;


	c) Les impositions, autres que celles incombant normalement à l'occupant, perçues, à raison desdites propriétés, au profit des collectivités territoriales, de certains établissements publics ou d'organismes divers, à l'exception de la taxe annuelle sur les locaux à usage de bureaux, les locaux commerciaux et les locaux de stockage perçue dans la région d'Ile-de-France prévue à l'article 231 ter ;

	d) Les intérêts de dettes contractées pour la conservation, l'acquisition, la construction, la réparation ou l'amélioration des propriétés, y compris celles dont le contribuable est nu-propriétaire et dont l'usufruit appartient à un organisme d'habitations à loyer modéré mentionné à l'article L. 411-2 du code de la construction et de l'habitation, à une société d'économie mixte ou à un organisme disposant de l'agrément prévu à l'article L. 365-1 du même code ;

	e) Les frais de gestion, fixés à 20 € par local, majorés, lorsque ces dépenses sont effectivement supportées par le propriétaire, des frais de rémunération des gardes et concierges, des frais de procédure et des frais de rémunération, honoraire et commission versés à un tiers pour la gestion des immeubles ;

	e bis) Les dépenses supportées par un fonds de placement immobilier mentionné à l'article 239 nonies au titre des frais de fonctionnement et de gestion à proportion des actifs mentionnés au a du 1° du II de l'article L. 214-81 du code monétaire et financier détenus directement ou indirectement par le fonds, à l'exclusion des frais de gestion variables perçus par la société de gestion mentionnée à l'article L. 214-61 du même code en fonction des performances réalisées.

	Les frais de gestion, de souscription et de transaction supportés directement par les porteurs de parts d'un fonds de placement immobilier mentionné à l'article 239 nonies ne sont pas compris dans les charges de la propriété admises en déduction ;

	f) pour les logements situés en France, acquis neufs ou en l'état futur d'achèvement entre le 1er janvier 1996 et le 31 décembre 1998 et à la demande du contribuable, une déduction au titre de l'amortissement égale à 10 % du prix d'acquisition du logement pour les quatre premières années et à 2 % de ce prix pour les vingt années suivantes. La période d'amortissement a pour point de départ le premier jour du mois de l'achèvement de l'immeuble ou de son acquisition si elle est postérieure.
				\end{quote}


			Elle n'est pas limitatrice (article 13 )

			\begin{itemize}
				\item Sur les dépenses d'améliorations ... (interruption)
			\end{itemize}

			\subsubsection{Régime du Micro-foncier}

			deduction d'impot = somme deduite de la base impossable
			reduction d'impot = somme retirée de l'impot dû
			credit d'impot = reduction d'impot qui est rembourse en cas d'impot négatif

			C'est une option à faire chaque année.

				\paragraph{Champ d'application}

				Moins de \montant{15000} de recette


				\paragraph{Modalités d'imposition}

				Extremement simple : ensemble des recettes perçues deduite de l'abbatement forfaitaire de \pourcent{30}
				C'est l'administration elle même qui réalise l'abbatement.


	\subsection{Imputation des déficits fonciers}

		Dispositif prévu à l'\articleDu{156 par. 1\ier 3emme}{\cgi}

		Abus avocats fiscalistes


		\subsubsection{Principes}

			\begin{enumerate}
				\item Les deficits fonciers sont imputables sur le revenu global que dans la limite annuelle de \montant{10700}\footnote{\articleDu{156}{\cgi}}.
				\item Les biens qui ont permis d'être en situation doivent être loué jusqu'au 31 décembre de la 3\ieme{} qui suit celle de l'imputation.
				\item La fraction de déficit supérieur au \montant{10700} est reportable sur les dix années suivantes, y compris les interets d'emprunt.
					\begin{exemple}
						Je peux étaler les 300 sur les
					\end{exemple}
				\item Si le contribuable a un revenu global inférieur à \montant{10700}, l'excedent de déficit est imputable sur les revenus gloax des six années suivantes.
			\end{enumerate}

		\subsubsection{Exceptions}

			Cas où le déficit ...
			\begin{enumerate}
				\item Les dépenses autres que les interets d'emprunt réalisées en vue de la resaturation complete d'un immeuble bâti situé en secteur sauvegardé ou assimilé. C'est ce que l'on appelle la loi << Malraux >>, même si ce n'est plus codifié sous la loi de 1962.
				\item Les charges afférentes aux immeubles historiques ... logement (36').
				\item Les dépenses de préservation et d'améliration du patrimoine naturel, autres que les interets d'emprunts
			\end{enumerate}


		\subsubsection{Imputation des déficits fonciers, Décision de rescrit 22-6-2010 \no 2010/35}

		\begin{casPratique}{%
			Cas pratique n°12 : Calcul d’impôt sur le revenu et de prélèvements sociaux.}
			Une personne célibataire a pour 2017 un revenu global de 115.000 €, hors un revenu foncier net de 53.000 €.

			Elle dispose d’un déficit foncier de 2012 non utilisé de 3.000 €

			Combien cette personne paiera-t-elle d’impôt sur le revenu et de prélèvements sociaux au titre de ses revenus 2017 ?

			Quid si cette personne est pacsé (sans modification)
		\end{casPratique}
		Le régime général. Le régime microfoncier si < 15000
		Son revenu foncier est supérieur à \montant{15000}
		Donc régime général.

		Régime d'mposietion = recette - dépense.
		Recette = 53
		Pas de dépense

		Imputation de déficit foncier.
		On fait l'hypothèse que les biens seront loué jusqu'au 31 décembre 2020.
		La fraction est reportable 10 ans. 2012 < 10 ans
		-< reportable
		Dans la limite de \montant{10700}
		\montant{3000} < \montant{10700}

		Donc revenu global = \montant{165000}

		Celibataire donc quotient familial -=1
		TRANCHE DU REVENU 2019
(Quotient familial)	TAUX D’IMPOSITION
(TMI)
		\begin{tabular}
		jusqu'à 10 064 euros	& 0 & 0 \\
			de 10 064 à 27 794 euros	14 -> 2482,20
			de 27 794 à 74 517 euros	30 %
			de 74 517 à 157 806 euros 	41 %
			Supérieur à 157 806 euros 	45 %
		\end{tabular}


Total = 53884,89

Prélèvements sociaux
- CSG									9,2 ->  erreur 15180
- CRDS								0,5 ->  825
- Prélèvement de solidarité-> 7,5% 12375

Montant total = \montant{62465}




\section{Régime d’investissement \nom{Duflot} - \nom{Pinel}}

	A l'origine

	50' \articleDu{}{\cgi}

	\subsection{Champ d’application}


		\subsubsection{Personnes concernées}

			Contribuable personne physique domicilié fiscalement en France. Soit dirceyement, soit via une SCI si celle-ci est translucide, soit par une SCPI (qui est dans tous les cas translucide).

		\subsubsection{Investissements concernés}

			\begin{itemize}
				\item Acquisition d'un immeuble
				\item Acquisition de part d'une SCPI
			\end{itemize}

			\paragraph{Acquisition d'un immeuble}

				Peu importe la façon dont l'immeuble est acquit.

				La difficulté intervient lorsque la propriété est démembré. Le régime \Duflot ne s'applique pas à un bien démembré.

				De plus, l'immeuble doit avoir été acquis à titre onéreux\footnote{Un apport à une société est considéré comme un apport à titre onéreux. Un apport à une SCI permet donc contourner cette difficulté.}.

				\begin{enumerate}[label=\alpha*. ]
					\item \textbf{Acquisitions de logements neufs achevés} Il faut entendre immeuble neuf innocupé, même par le propriétaire.
					\item \textbf{Acquisitions de logements en l'état futur d'achèvement}
					\item \textbf{Acquisition d'un logement en vue de sa réhabilitation} C'est un peu technique. Le logement doit ne pas répondre aux caractèristiques de décence, et que vous les travaux les rendent décents. Il y a une procédure un peu complexe. Cf. \vref{tab:decenceEtPerformanceDuLogement}.
					\item \textbf{Acquisition d'un logement réhabilité}
					\item \textbf{Acquisition d'un local que le contribuable transforme en habitation}
					\item \textbf{Acquisition d'un logement issu de la transformation d'un local affecté à un autre usage}
					\item \textbf{Acquisition d'un logement qui a fait ou qui fait l'objet de travaux concourant à la production ou à la livraison d'un immeuble neuf au sens de la TVA}
					\item \textbf{Logement que le contribuable fait construire}
					\item \textbf{Acquisition d'un local inachevé}
				\end{enumerate}

				\begin{table}[h]
					\centering
					\label{tab:decenceEtPerformanceDuLogement}
					\begin{tabular}{0.8\linewidth}{lXX}
						\toprule
						  & Etat du logement avant les travaux &	Etat du logement après les travaux \\
						\midrule
						Décence du logement	& Au moins 4 des 15 caractéristiques de décence ne sont pas respectées (Arrêté du 19 décembre 2003 art. 3) &	L'ensemble des 15 caractéristiques de décence est réuni (Arrêté du 19 décembre 2003 art. 3) \\
						Performance technique	& Au moins 6 des 12 performances techniques ne sont pas respectées (Arrêté du 19 décembre 2003 art. 4) &	L'ensemble des performances techniques est réuni (Arrêté du 19 décembre 2003 art. 4) \\
						\bottomrule
					\end{tabular}
					\caption{« décence et performance du logement » }
				\end{table}


			\paragraph{Souscription de parts de SCPI}

				C'est le cas d'un contribuable qui ne veut pas être concerné par les problèmatiques de maitrise d'ouvrage ... L'argent doit être investi dans les 18 mois qui suivent la levée d'argent.


		\subsubsection{Caractéristiques des immeubles}



			\paragraph{Immeuble à usage de logement}

			cas des immeubles 3/4

			\paragraph{Situation de l'immeuble}

				Deux possibilité dans le régime selon que l'investissement soit réalisé en métropole ou non.

				\begin{enumerate}[label=\textbf{\alpha*.}]
					\item \textbf{Investissements réalisés en métropole.} article O 4 bis c de l'annexe 4

					\item \textbf{Investissements réalisés outre-mer.} Cela s'applique partout.
				\end{enumerate}


			\paragraph{Performance énergétique globale des logements}

				L'idée est de justifier d'un certain nombre de performance énérgetique.

				\begin{enumerate}[label=\textbf{\alpha*.}]
					\item \textbf{Logements situés en métropole} Le niveau de performance global fixé par l'\article46 AZA -a de l'annexe III

					\item \textbf{Investissements réalisés outre-mer.} Cela s'applique partout.
				\end{enumerate}

				\subparagraph{a. Logements situés en métropole}

				\subparagraph{b. Investissements réalisés outre-mer}


	\subsection{Conditions d’application de la réduction d’impôt}

		\subsubsection{Engagement de location}

			3 conditions

			\paragraph{Engagement du propriétaire du logement} pas du même foyer fiscal.

			\paragraph{Contenu de l'engagement de location} Ce contenu est fixé par le décret 2013-1235 du 23/12/2013. Cet engagement doit faire mention du loyer et des ressources du locataires. Un plafond d eloyer et un plafon de ressource.

			\paragraph{Constatation de l'engagement de location} Il faut être trés précis, car l'idée est de dire voilà c'est parti je m'engage dan sle duflot.

			lorsque le personne physique. Au moment du depot de la déclaration de revenu de l'année au titre de laquelle le fait générateur est intervenu. Cela depend de la forme.

			Si vous achetez un immeuble neuf pour le louer. C'est l'année de l. Si achat 2019

			On peut donc decalfer un duflot alors que l'immeuble n'est pas encore loué.

			Deuxième cas CSI translucide. au moment d ela déclaration de résultat de l'année au titre de laquelle le fait générateur est intervenu.

			Trisème cas SCPI. pris dans la déclaration de l'année de souscription des parts.


		\subsubsection{Engagement de conservation des parts}

			Ne concerne

			\begin{enumerate}
				\item Société autre que SCPI (société translucide) Engagement à conserver jusqu'au terme de l'engagement pris par la société.
				\item SCPI Engagement à conserver jusqu'au terme de l'engagement pris par la société.
			\end{enumerate}

			\begin{casPratique}{Cas pratique n°13}
				Un contribuable réalise une souscription au capital d'une SCPI le 1er juin 2013 avec engagement de conservation des titres pendant 9 ans.

				Cette souscription est affectée au financement de cent trente logements.

				Les contrats de location prennent effet au 18 novembre 2013 pour le premier de ces logements et au 1er janvier 2014 pour le dernier.

				Jusqu’à quelle date le contribuable doit-il conserver ses parts pour bénéficier du dispositif ?
			\end{casPratique}

			Dernier logement pour lequel affecte + 9 ans
			En l'espèce 1/1/2014 + 9 ans => 31/12/2022


			\subsubsection{Conditions de mise en location}

				Ces conditions sont au nombre de cinq :
				\begin{enumerate}
					\item \textbf{Délai de mise en location.} Le logement doit être donné en location nue dans un délai de douze mois à compter de son achèvement ou de son acquisition si celle-ci est postérieure (CGI art. 199 novovicies, III).
					\begin{table}
						\centering
						\label{tab:}
						\begin{tabularx}{0,75\linewidth}{}
							\toprule
								& \textbf{Délai de mise en location} \\
							\midrule
								Acquisition d'un logement neuf achevé & \\
								Acquisition d'un logement issu de la transformation d'un local affecté à un usage autre que l'habitation qui entre ou non dans le champ de la TVA & \\
								Acquisition d'un logement réhabilité & \\
								Acquisition d'un logement qui a fait l'objet de travaux concourant à la production ou à la livraison d'un immeuble neuf au sens de la TVA & Douze mois qui suivent la date d'acquisition \\
								Acquisition d'un logement en vue de sa réhabilitation & \\
								Acquisition d'un logement qui fait l'objet de travaux concourant à la production ou à la livraison d'un immeuble neuf au sens de la TVA	Douze mois qui suivent la date d'achèvement des travaux & \\
								Acquisition d'un logement en état futur d'achèvement & \\
								Acquisition d'un logement qui fait l'objet de travaux concourant à la production ou à la livraison d'un immeuble neuf au sens de la TVA & Douze mois qui suivent la date d'achèvement du logement \\
								Acquisition d'un local que le contribuable transforme en logement	& \\
								Acquisition de locaux inachevés	& \\
								Construction d'un logement par le contribuable & \\
							\bottomrule
						\end{tabularx}
						\caption{}
					\end{table}
					\item \textbf{Durée de location} 6 ans reconductible 2x 3 ans ou 9 ans reconductible 1 fois
					\item \textbf{Affectation des logements} Non meublé, à usage d'habitation principale du locataire. La location doit être effective et continu pendant toute la période de location. Ce sont des points difficilement maitrisable.

					Si le locataire donne congé, il faut remetre immédiatement en location, et la période de vacance ne doit pas exceder 12 mois
					\item \textbf{Qualité du locataire} Personne physique respectant des plafonds de ressources
					\item \textbf{Location à un organisme public ou privé en vue de sa sous-location}
				\end{enumerate}


			\subsubsection{Plafonds de loyer}


				\paragraph{Investissements réalisés en métropole}

					\begin{enumerate}[label=\alpha*.]
						\item \textbf{Plafonds de loyer mensuel par mètre carré} ???? \cgi. Pour chacune des aires il est défini un plafonds.
						Un calculateur de plafonds de loyer est disponible sur le site internet du ministère de l'égalité des territoires et du logement (www.territoires.gouv.fr).
						\item \textbf{Surface à prendre en compte} C'est la surface habitable plus la moitié des surfaces annexes, dans la limite de \surface{8}\footnote{les surfaces de garages ne sont pas à prendre en compte}.
					\end{enumerate}

					\begin{casPratique}{%
						Une maison individuelle de 90 m2 de surface habitable comporte des combles aménageables d'une surface de 50 m2 et un sous-sol de 90 m2 dont une partie est accessible au stationnement d'un véhicule.

						Quelle surface faut-il retenir pour l’appréciation du plafond de loyer ?}

						Surface habitable = 90
						Surface annexe 50m. mais plafonds à 8
						Administration fiscale enleve 12m2 pour les stationnements
						Donc 98
					\end{casPratique}

				\paragraph{Investissements réalisés outre-mer} Les plafonds sont, depuis peu, identiques quel que soit le lieu et égal à \montant{9,88}.


				\paragraph{Dispositions communes : le coefficient multiplicateur} L'idée est de tenir compte de la surface. Le plafond est multiplié par le coefficient.




			\subsubsection{Plafonds de ressources des locataires}

				Différents
				\begin{enumerate}[label=]
					\item \textbf{Investissements réalisés en métropole} ??? annexe III \cgi (27' 31). Ces ressources sont celles de l'anté pénultimème année.

					Pour les baux conclus en 2013 en métropole, les plafonds annuels de ressources des locataires ont été fixés par le décret 2012-1532 du 29 décembre 2012 comme indiqué au tableau.
					\item \textbf{Investissements réalisés outre-mer}
					C'est tout à fait équivalent. Art 2 ter decies f de l'annexe III du \cgi. Pour les baux conclus à compter du 8 juin 2013 dans les DOM, les plafonds annuels de ressources des locataires ont été fixés comme suit par le décret 2013-474 du 5 juin 2013 :
					Ils sont commun à l'ensemble des collectivité d'outre-mer.
					\item \textbf{Dispositions communes}
					\begin{enumerate}[label=\alpha*]
						\item \textbf{Obligations déclaratives} Il faut joindre à la declaration 2042. Premier problème : en cs de décalage entre l'acquisition et la location. Dans ces cas il faut faire la déclaration, et vous complétez l'année suivante.
						\item \textbf{Ressources à prendre en compte} Ce sont les ressources de l'antépéutième année. Le revenu est le revnu fiscal de référence.
						\item \textbf{Location à un organisme public ou privé en vue de sa sous-location} Ce sont à ces organismes de vérifier les conditions de secours. Il faut passer une convention pour leur faire supporter les conséquences d'un éventuel manquement.
					\end{enumerate}
				\end{enumerate}

% 1. Investissements réalisés en métropole
% 				\paragraph{Investissements réalisés en métropole}
%
% 2. Investissements réalisés outre-mer
% 				\paragraph{Investissements réalisés outre-mer}
%
% 3. Dispositions communes
% 				\paragraph{Dispositions communes}
%
%
% a. Obligations déclaratives
% 					\subparagraph{Obligations déclaratives}
%
% b. Ressources à prendre en compte
% 					\subparagraph{Ressources à prendre en compte}
%
% c. Location à un organisme public ou privé en vue de sa sous-location
% 					\subparagraph{Location à un organisme public ou privé en vue de sa sous-location}


	\subsection{Modalités d’application de la réduction d’impôt}

		\subsubsection{Base de la réduction d'impôt}

				\paragraph{Acquisition de logements}

					\begin{enumerate}[label=\alpha*.]
						\item \textbf{Détermination de la base de la réduction d'impôt} C'est le prix d'acquisition ou le prix ??? (44').
						\item \textbf{Plafond de la base de la réduction d'impôt} 199 novovicies V du \cgi au titre d'une ne peut pas exceder 300000
							\begin{quote}
								<<
							\end{quote}

							On ne peut inscrire que deux logements par ans et par personne.
						\item \textbf{Plafond de prix de revient par mètre carré de surface habitable} 46 A z A de l'annexe \III du \CGI 5500euro/m2

						\item \textbf{Articulation entre les plafonds} Le plafond de (5500euro) ne permet pas de dépacer le montant de plafond total. Il y a donc double plafonnement: on garde à chaque fois le plus abs des plafonds.
					\end{enumerate}

%a. Détermination de la base de la réduction d'impôt
% 					\subparagraph{Détermination de la base de la réduction d'impôt}
%
% b. Plafond de la base de la réduction d'impôt
% 					\subparagraph{Plafond de la base de la réduction d'impôt}
%
% c. Plafond de prix de revient par mètre carré de surface habitable
% 					\subparagraph{Plafond de prix de revient par mètre carré de surface habitable}
%
% d. Articulation entre les plafonds
% 					\subparagraph{Articulation entre les plafonds}

			\begin{casPratique}{%
				Un contribuable acquiert un logement pour la somme de 300 000 €.

				Le prix de revient par mètre carré de surface habitable est de 7 500 €/m2 et la surface habitable du logement est de 40 m2.

				Quel montant doit-on retenir pour le calcul de la réduction d’impôt ?
				}
				plafond 2 : 5 500 €/m2 x 40 = 220 000
				plafond 1  : 300000

				Donc : 220 000
			\end{casPratique}

			\begin{casPratique}{%
				Un contribuable acquiert un logement pour la somme de 525 000 €.

				Le prix de revient par mètre carré de surface habitable est de 7 500 €/m2 et la surface habitable du logement est de 70 m2.

				Quel montant doit-on retenir pour le calcul de la réduction d’impôt ?
				}
				prix de base = 525000

				plafond 2  : 5 500 €/m2 x 70 = 385000
				plafond 1  : 300000

				Donc : 300000

			\end{casPratique}

			\begin{casPratique}{%
				Un contribuable acquiert deux logements au titre d'une même année.

				Le premier logement est acquis pour la somme de 203 000 €.

				Son prix de revient par mètre carré de surface habitable est de 5 800 €/m2 et sa surface habitable est de 35 m2.

				Le deuxième logement est acquis pour la somme de 180 000 €. Son prix de revient par mètre carré de surface habitable est de 6 000 €/m2 et sa surface habitable est de 30 m2.

				Quel montant doit-on retenir pour le calcul de la réduction d’impôt ?
				}

				prix de base  : 5 800 €/m2 x 35 m2 + 6 000 €/m2 x 30 m2 = 203000 + 180000 = 383000

				plafond 2 : 357500
				plafond 1 : 300000

				Donc : 300000
			\end{casPratique}


%2. Souscription de parts de SCPI
				\paragraph{Souscription de parts de SCPI}

					\begin{enumerate}[label=\alpha*.]
						\item \tetxtbf{Plafond de la base de réduction d'impôt.} C'est 100\% de la souscription, plafonné à \montant{300000}.
						\item \tetxtbf{Acquisition en indivision.} Le montant s'apprécie pour la part indivise de chacun des indivisaires.
					\end{enumerate}
% a. Plafond de la base de réduction d'impôt
% 					\subparagraph{Plafond de la base de réduction d'impôt}
%
% b. Acquisition en indivision
% 					\subparagraph{Acquisition en indivision}


%3. Acquisition d'un logement et souscription de parts de SCPI au titre d'une même année
				\paragraph{Acquisition d'un logement et souscription de parts de SCPI au titre d'une même année}

					Le
					\begin{casPratique}{%
						Au titre d'une même année d'imposition, un contribuable acquiert un logement et réalise une souscription de parts de SCPI.

						Le logement est acquis pour la somme de \montant{350000}.

						Son prix de revient par mètre carré de surface habitable est de 7 500 €/m2 et sa surface habitable est de 50 m2.

						Le montant de la souscription de parts de SCPI s'élève à 50 000 €.

						Quel montant doit-on retenir pour le calcul de la réduction d’impôt ?
						}
						Montant de base acquisition :
						- prix \montant{350000}
						- prix de revient 7 500 €/m2 x 50 m = \montant{375000}
						=> \montant{350000}
						Plafond 1 = \montant{300000}
						Plafond 2 = \montant{275000}

						Donc \montant275000

						SCPI = \montant50000
						Plafond \montant300000
						Donc \montant50

						Somme \montant{275} + \montant{50} : \montant{325}
						Plafond dans tous les cas \montant{300}

						Donc \montant{300000}
					\end{casPratique}


%B. Taux de la réduction d'impôt
			\subsubsection{Taux de la réduction d'impôt}
				ces taux dépendent du lieu et de la durée.

				\begin{enumerate}
					\item metropole
					\begin{enumerate}
						\item 6 ans 12\% reconductible pour 3 ans auquel cas 6\% de plus et 3\% étalable sur 3 ans
						\item 9 ans 18\% étalé sur 9 ans, prolongeable de 3 ans 3\%
					\end{enumerate}
					Dans tous les cas, si on atteint
					\item outre mer
					\begin{enumerate}
						\item 6 ans 23\% sur 6 ans
						\item 9 ans 29\% étalable sur 9 ans
					\end{enumerate}
				\end{enumerate}

%C. Modalités d'imputation de la réduction d'impôt
			\subsubsection{Modalités d'imputation de la réduction d'impôt}

				Dès lors que le logement esta cquis, vous déclencher le bénéfice de la déclaration d'impot.

%D. Limitation du nombre d'investissements
			\subsubsection{Limitation du nombre d'investissements}

				2 investissement par an

				\begin{casPratique}{%
					Un contribuable acquiert en 2017 deux logements en l'état futur d'achèvement situés en métropole pour un prix total de 350 000 €.

					Ce même contribuable acquiert, en 2018, deux autres logements en l'état futur d'achèvement situés outre-mer pour un prix total de 320 000 €.

					L'achèvement de ces quatre logements, qui correspond au fait générateur de la réduction d'impôt, intervient au cours de l'année 2019. Le prix de revient par mètre carré de surface habitable des logements pris en exemple est réputé être inférieur au plafond de 5 500 €

					Quel sera le montant de la réduction d'impôt imputable au maximum ?
					}
					Pinel possible ? oui
					On peut prendre les 4 car 2 par ans.
					Chaque année, le plafonds est de \montant{300000}.
					metropole
					\begin{enumerate}
						\item 6 ans 6000
						\item 9 ans 6000
						\item 12 ans
					\end{enumerate}
					outre mer
					Plafond total de \montant{300000}
					\begin{itemize}
						\item 6 ans
						\item 9 ans
					\end{itemize}

					Attention plafonnement des niches fiscales à \montant{10000}
				\end{casPratique}
				\begin{casPratique}{%
					Deux personnes physiques constituent à parts égales une société civile non soumise à l'impôt sur les sociétés.

					Cette société acquiert cette année deux logements neufs achevés situés en métropole pour un prix de 240 000 €.

					Quel sera le montant de la réduction d'impôt imputable par chaque associé au maximum ?
					}
						120000
						plafond
				\end{casPratique}
				\begin{casPratique}{%
					Deux personnes physiques constituent à parts égales une société civile, non soumise à l'impôt sur les sociétés.

					Cette société acquiert cette année trois logements neufs achevés situés en métropole pour un prix total de 390 000 € (150 000 € pour le premier logement, 140 000 € pour le deuxième et 100 000 € pour le dernier).

					Quel sera le montant de la réduction d'impôt imputable par chaque associé?
					}
					1965Plafnd
				\end{casPratique}

%E. Règles de non-cumul
			\subsubsection{Règles de non-cumul}


F. Articulation avec d'autres dispositifs
			\subsubsection{Articulation avec d'autres dispositifs}


			Cas pratique n°22 :

			Un contribuable acquiert cette année un logement en vue de sa réhabilitation pour un montant de 150 000 €.

			Le montant des travaux de réhabilitation s'élève à 50 000 €, qui pour partie ouvrent droit pour le contribuable au crédit d'impôt prévu à l'article 200 quater du CGI pour un montant (en droits) de 1 200 €.

			Quel sera le prix de revient à prendre en compte pour le calcul de la réduction d’impôt ?


	\subsection{Obligations}
IV- Obligations
		\subsubsection{Logements acquis directement par le contribuable}

A. Logements acquis directement par le contribuable

		\subsubsection{Logements acquis par l'intermédiaire d'une société non soumise à l'IS}
B. Logements acquis par l'intermédiaire d'une société non soumise à l'IS

1. Obligations des sociétés
2. Obligations des associés

		\subsubsection{Souscription de parts de SCPI}
C. Souscription de parts de SCPI

			\paragraph{Obligations des sociétés}
1. Obligations des sociétés

			\paragraph{Obligations des associés}

2. Obligations des associés
