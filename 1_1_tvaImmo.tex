\chapter{La TVA immobilière}

L’impôt qui parait le plus compliqué : la TVA immo


En 1917 : invention de l’IR et de la taxation en cascade (\pourcent{2}). Taxation en cascade jusqu’en 1936. Car a pour conséquence la concentration, la suppression des intermédiaires et le renchérissement des produits. Taxation unique de \pourcent{6}. En 1945, plus de système fiscal ; Maurice \nom{Laurè} en 1953 Restauration de la taxation en cascade Son idée est de ne pas taxer les entreprises car créent de la richesse. Invente le fait de récupérer en aval ce qu’elles payent en amont.


Image chupa chups


Bâtonnet -> Bâtonnet + Boule de sucre -> Bâtonnet + Boule de sucre + Emballage


A vérifier la différence entre entreprise et société. Toutes les entreprises ne sont pas des entreprises, toutes les sociétés sont des entreprises.


Définition communautaire : impôt général sur la consommation, proportionnel au prix des biens et des services, perçu à chaque stade du processus de production ou de distribution, mais seulement sur la valeur ajoutée à chacun de ces stades grâce au mécanisme de la déduction de la taxe acquittée en amont par le producteur et répercutée en aval sur le consommateur


15 03 1963, immeuble assujetti à la TVA --- la TVA vient remplacer les droits d’enregistrement. La TVE est portée par l’acquéreur.


6ieme directive européenne 1977 système commun de TVA. Conseil d’État 2006 explique que si la France ne retranscrit pas une directive … l’administré peut se prévaloir du droit communautaire (arrêt Nicolo).

Entre 2006 et 2009 les petits malins n’ont pas payé… 2/12/2009 proposition Wasman … affaire du Grand Emprunt 10/03/2010 … pas de régime transitoire … nego des notaires … aujourd’hui 2x régime terminé. Donc aujourd’hui même TVA que l’on vende des sucettes ou des immeubles.



\section{Champ d’application de la TVA immobilière : les opérations réalisées par les assujettis dans le cadre de leur activité économique}
	
	Assujetti : 256 A CGI << c’est la personne qui effectue de manière indépendante une activité économique quel que soit son statut juridique, sa situation au regard des autres impôts et la forme ou la nature de son intervention >>

	
	L'assujeti est donc une personne. Indépendant = pas de lien de subordination

	
	Activité économique = activité dont on tire un profit à court terme et répétitif


	\subsection{Les opérations taxables de plein droit}
		
		\subsubsection{Notion de terrain à bâtir}
			
			1\iere{} difficulté : la définition. Elle est différente pour le champ de la TVA de celle du droit de l’enregistrement.

			Terrain à bâtir : << les terrains sur lesquels des constructions peuvent être autorisées en application d’un PLU ou d’un autre doc d’urbanisme >>.
			C’est une définition objective, il n’y a aucune déclaration à faire. L’administration considère que le terrain est constructible si il n’y pas de dépense substantielle pour la desserte.

			
			Un immeuble bâti, mais voué à la démolition reconstruction est-il un terrain à bâtir ? Arrêt Don Bosco CJCE C46108 du 17/11/2009 : oui si il incombe au vendeur les frais de démolition. Sinon, non.

		
		\subsubsection{Notion d’Immeubles neufs }
		
			Il faut être vigilant puisque la notion << d'immeuble neuf >> diffère de celle d'autre champs, comme celui de loi \nom{Pinel}.
			
			Un immeuble est considéré comme neuf lorsqu'il a moins de 5 ans.
			Il peut s'agir :
			\begin{itemize}
				\item soit d'une construction nouvelle ;
				\item soit d'un immeuble ancien mais ayant fait l'objet de travaux de rénovation dont l'importance est telle qu'ils rendent le bien à l'état neuf.
			\end{itemize}
			
			Les travaux sont considérés fiscalement comme rendant le bien à l'état neuf, quand ils portent sur l'un des quatre cas suivante :
			\begin{enumerate}
				\item la majorité des fondations ;
				\item la majorité des éléments hors fondations déterminant la résistance et la rigidité de l'ouvrage ;
				\item la majorité de la consistance des façades (hors ravalement) ;
				\item l'ensemble des éléments de second œuvre dans une proportion fixée par un décret en Conseil d'État (\pourcent{50}). C'est le cas le plus difficile à apprécier. En particulier, les marchands de biens cherchant à éviter la TVA qui réhabilite logement par logement un immeuble.
			\end{enumerate}
			
		\subsubsection{Notion de droits assimilés}
			
			\begin{itemize}
				\item les droits réels immobiliers (nu propriété, usufruits, \etc) ;
				\item les droits relatifs aux promesses de vente (malgré le caractère interdit de cette pratique le grâce à la théorie du réalisme du droit fiscal) ;
				\item les ventes de part d'intérêt et d'action dont la possession entraine un droit de jouissance --- appartements de sport d'hivers ou au bord de la mer par exemple ;
			\end{itemize}
		
		\subsubsection{Livraisons à soi-même}
			
			Il s'agit d'un mécanisme assez complexe à appréhender, dans lequel un professionnel s'auto-approprie une partie de sa production.
			
			Il date des années 1970 : Renault parvenait à réduire ses coûts de production en construisant lui-même ses chaines de montage, échappant ainsi à la TVA. Le gouvernement a alors décidé de taxer comme si Renault achetait à une entreprise tierce. Il est très utilisé notamment pour les brasseries et les restaurants, dans lesquels le personnel déjeune au sein de l'établissement.
			
			 \medskip \href{http://bofip.impots.gouv.fr/bofip/134-PGP.html?identifiant=BOI-TVA-CHAMP-10-20-20-20160302}{\bfseries BOI-TVA-CHAMP-10-20-20-20160302}
			\begin{quote}
				<< {\itshape La livraison à soi-même est l'opération par laquelle une personne obtient, avec ou sans le concours de tiers, un bien meuble ou immeuble ou une prestation de services à partir de biens, d'éléments ou de moyens lui appartenant.} >>
			\end{quote}
			
			Techniquement, on inscrit la livraison dans sa déclaration mensuelle de TVA, et ensuite il y a modification du bilan : l'immeuble passe du stock (actif) en immobilisation (passif). Vous avez le droit de vous livrez à vous même dans un laps de temps limité : {\bfseries jusqu'au 31 décembre de la deuxième année qui suit l'achèvement de l'immeuble}\footnote{Il n'est pas obligatoire de conserver la valeur au bilan, mais avec des effets sur le résultat}.
	
	\subsection{Les opérations exonérées taxables sur option}
		
		Pour un assujetti, un professionnel de l'immobilier, demander à être taxer alors qu'on est exonéré n'est pas assimilable à du masochisme. Cela permet de récupérer la TVA d'amont. Dans la plupart des cas, cela est souhaitable dans le cas de mauvaise affaire.
		
		\medskip Quels sont ces cas d'opérations exonérées mais que l'on peut demander à être taxées ?
		\begin{enumerate}
			\item livraison de terrain non à bâtir (Terrain NAB) ;
			\item livraison d'immeuble achevé depuis plus de 5 ans ;
			\item les baux conférant un droit réelle immobilier --- cas prévu à l'\articleCGI{261-D-1-bis}.
		\end{enumerate}
	Cela permet de récupérer la TVA d'amont, dans les cas où elle est supérieure à la TVA d'aval, et ainsi d'avoir une perte moins mauvaise.
	
\section{Les règles d’imposition}

	\subsection{Base d’imposition}
		
		La base d'imposition est la grande difficulté du régime car il y en a deux possibles :
		\begin{enumerate}
			\item sur le prix total ;
			\item sur la marge.
		\end{enumerate}
		
		\paragraph{TVA sur le prix}
			
			La TVA porte sur le prix total dès lors Il y a lorsque l'immeuble à ouvert droit à déduction de TVA lors de son acquisition.
			
			\begin{exemple}
				Mr S. est un professionnel de l'immobilier. Mr B est également ...
				
				Mr M. n'est pas un professionnel de l'immobilier, il ne peut rien déduire.
				
				Si Mr B n'arrive pas à revendre dans les 5 ans, je sors du champs de la TVA. 
			\end{exemple}
			
		\paragraph{TVA sur la marge}
			
			La TVA porte dur la marge dans les autres cas, donc quant le bien vendu n'a pas ouvert droit à déduction de TVA lors de son acquisition.
			
			\medskip La marge est la différence entre le prix de vente et le prix d'acquisition.
			Vous ne pouvez pas tenir compte de travaux.
	
	\subsection{Fait générateur et exigibilité de la taxe}
		
		Trois cas.
		
		\subsubsection{Livraisons d’immeubles}
			
			Il n'y a aucune difficulté. L'immeuble est livré au moment du transfert de propriété. Sauf mention express différentes, c'est la signature de l'acte notarial.
			
			Si le paiement est échelonné, la TVA n'en reste pas moins collectée en une fois.
			
		\subsubsection{Ventes d’immeubles à construire}
			
			C'est une exception au principe : la TVA est appelée au fur et à mesure, en application de l'\articleCGI{269-2-a-bis}.
			
		\subsubsection{Livraisons à soi-même}
			
			C'est la livraison proprement dite : dès lors que vous faites le choix. C'est au moment de votre en application de l'\articleCGI{269-1-b}.
			
			pour mémoire 1/12 etc ...
			
	
	\subsection{Redevable de la TVA}
		
		Il s'agit du vendeur, depuis la grande réforme de 2010.
		
	
	\subsection{Taux de la taxe}
		
		Il y a plusieurs taux :
		
		\begin{center}
			\begin{tabular}{ll}
				\toprule 
				\textbf{taux} &  \\ 
				\midrule
					normal & \pourcent{20} \\
					réduit & \pourcent{10} \\
					<< super réduit >> & \pourcent{5,5} \\
				\bottomrule
			\end{tabular}
		\end{center}

		
	
	\subsection*{Cas pratique n°1}
	
		\begin{enonce}
			Rédiger un tableau des cas d’assujettissement à la TVA immobilière.
		\end{enonce}
	
	\subsection*{Cas pratique n°2}
	
		\begin{enonce}
			La société A exerce une activité d’achat-vente de biens immobiliers.  Elle est assujettie pour cette activité à la TVA. Elle envisage d’acheter une maison construite en 1950 et situé sur un terrain constructible de \surface{1 000}.

		
			\medskip Le particulier qui vend cette maison en veut \montant{500 000}.

			
			\medskip La société A souhaite couper ce terrain en deux lots de \surface{500} chacun, soit:

			\begin{itemize}
				\item un lot de \surface{500} comprenant la maison ;

				\item un lot de \surface{500} nu.

			\end{itemize}
			
			\medskip La société A a déjà trouvé deux acquéreurs pour ces biens :

			\begin{itemize}
				\item un professionnel de l’immobilier pour le lot comprenant la maison à \montant{600 000} ;

				\item particulier pour le terrain nu à \montant{300 000}.
			\end{itemize}

			
			\medskip La société A, informée de vos talents de fiscaliste, vous consulte pour déterminer la TVA applicable à l’opération et le bénéfice net qu’elle peut tirer de cette opération une fois les charges fiscales réglées.

		\end{enonce}
	
		Il faut raisonner par opération.
		
		\paragraph{1\iere{} opération} Concernant la vente d'un immeuble ancien par un non assujetti à un assujetti, la vente n'est pas soumise à TVA. Seuls les \dmto seront dus.
		
		Deux cas sont possibles :
		\begin{itemize}
			\item sans engagement de construire ;
			\item avec engagement de construire.
		\end{itemize}
		
		Les couts d'achats seront donc de : 
		
		\paragraph{2\ieme{} opération} Concernant la vente d'un immeuble ancien par un assujetti à un autre assujetti, la vente n'est pas soumise à TVA, sauf choix du vendeur. N'ayant pas de TVA amont, le vendeur n'a pas intérêt à se soumettre volontairement à la TVA.
		
		\paragraph{3\ieme{} opération} Concernant la vente d'un terrain à bâtir par un assujetti à un non assujetti. Comme le terrain n'a pas . Nous sommes donc dans le régime de la TVA sur marge. Le calcul de la marge doit tenir compte du fait que le prix de vente tient compte de la TVA à \pourcent{20} :
		
		\[ \textmd{Marge} = \frac{\textmd{Prix de vente (TTC)} -\textmd{Prix d'acquisition (HT)} }{1,2} \]
		
		\medskip Le prix d'acquisition Il y a trois possibilité pour ventiler le prix d'achat\footnote{Bien qu'il n'y ait pas de règles juridiques pour traiter ce cas} :
		\begin{itemize}
			\item proportionnellement aux surfaces de terrain vendues (suggérée par le \bofip) ;
			\item sur la base d'une estimation expert agréé par la Cour d'appel ;
			\item proportionnellement aux prix de revente.
		\end{itemize}
		Ayant les prix à la sortie, la solution << optimale >> est de ventiler proportionnellement aux prix de revente.
		
	
	\subsection*{Cas pratique n°3}
	
		\begin{enonce}
			La société A exerce une activité de construction et achat-vente de biens immobiliers. Elle souhaite acquérir un terrain à bâtir appartenant à M. X (particulier non assujetti à la TVA) sur la commune de Villemombles. Un accord a été trouvé aux termes duquel le terrain sera acheté \montant{1 200 000} nets vendeur. Le prix sera payé par l’attribution à M. X de trois appartements dans l’immeuble que la société A construira sur ce terrain.

			
			La société A vous consulte afin de connaître les régimes de TVA et de droits de mutation à titre onéreux applicables à cette opération. Elle vous interroge sur la neutralité financière de l’opération.

		\end{enonce}
	
		Nous sommes dans une opération de dation en paiement. Il y a en réalité fiscalement deux opérations : deux ventes croisées.
		
		La première opération est la vente par un non assujetti à un assujetti d'un terrain à bâtir. L'opération est exonérée de TVA.
		
		La deuxième opération est la vente d'immeuble neuf par un assujetti à un non assujetti. La vente est soumise à la TVA sur le prix total à \pourcent{20}, soit \montant{240 000}. \textbf{L'opération n'est donc pas neutre pour la société A}.
		
		Depuis que la TVA est due par le vendeur, les dations ne sont plus neutres. Il faut donc prévoir dans l'acte de bien préciser les montants HT et TTC.
		
		