% !TeX root = droitFiscalité.tex
\chapter{Les Taxes foncières}

	Il existe deux catégories de taxes foncières. Elles font paties des << quatre vielles >>.

	L'idée de base est intéressante et vient en droite ligne de la révolution de 1789. La Constituante instaure le versement volontaire et ce fut un échec.
	Puis vient l'impot sur les portes et fenetres. Ce qui a conduit à murer les-dites portes et fenetre, et instaurer une architecture fermée, aveugle sur le domaine public.

	\section{Taxe foncière sur les propriétés bâties}

		C'est un impot perçu au profit des communes (ou EPCI), de dpt et des regions, sauf en idf.

		Mecanisme assez complexe d'exonération.

		base impossable : valeur locative de l'immeuble, réduite d'un abbatement forfaitaer de 50\%

		Chaque \colloc* fixe le taux de la taxe. Strauss Khan a instaurer un mécanisme pour freiner les augmentation de taux de l'antepenultieme année.

		Le problème est la base de la valeur locative cadastrale. La dernière fois c'est en 1970.
		Deux mécanisme
		\begin{itemize}
			\item augmantation par la commune apres constatation de l'amelioration
			\item augmentation forfaitaire
		\end{itemize}
		En pratique les communes n'ont jamais augmenté la valeur locative cadastrale, probablement pour des considérations électorales.

		En 1981, le président \nom{Miterrand} a décidé de réviser les vlc.

		La loi de finance de 2020 prévoit de mettre en place une demande d'information des valeurs locatives.

	\section{Taxe foncière sur les propriétés non bâties}

		L'abbatement est seulement de 20\%.

		Système de perequation des taux.

		Possibilité d'exonération pour les agriculteurs exploitants.

		A la difference de la taxe, celle-ci est extremement lourde. En effet, en 1970 la terre avait une valeur bien supérieure à ce qu'elle ne vaut aujourd'hui. Il arrive que la taxe soit supérieure au loyer.

		Rapporte peu : (à vérifier)
		1 milliard
		40 million
		10 million
