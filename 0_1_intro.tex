INTRODUCTION

-	Présentation générale du système fiscal français et des grands problèmes de finances publiques ;

-	Présentation générale du système fiscal français et des grands problèmes de finances publiques ;


Qui paie ?


Tout le monde. La rév. De 1789 est une rev. Fiscale.


Pourquoi ?


Parcequ’on est obligé

Encadrement des paiements fait en 1992.

3 grandes catégories d’adm

ODAC … central

ODAL local

ODAS sécurité social


ODAC : en France c’est l’Etat et ses établissements publics

ODAL : en France : 13 régions, départements, communes, EPCI, leurs établissements publics

ODAS : 4 branches


Combien ?

Tout est fait pour que l’on ne sache pas : anesthésie fiscale


Adm	Recette	Dépense	Déficit	Dette

ODAC	320 M	509 M	189 M	1 855 M

ODAL	213 M	214 M	1 M	200 M

ODAS	498 M	535 M	37 M	250 M

Total	1 031 M	1 258 M	227 M	2 305 M


Les ODAL ont obligation de voter un budget en équilibre. 

En théorie on ne peut emprunter que pour investir, et dans ce cas l’emprunt est neutre. 

PIB = 2 283

Tx de pression fiscale = Prlvmnt obligatoir \ PIB ~ 45 %


USA 26 %

Japon 17 %

Allemagne 43 %

Suède 44%

Danemark 45 %


Tx de dépense publique ~ 55 %

USA 37

Allemagne 44 %

Suède 50 %

Danmak 55%


-	Définition du droit fiscal immobilier.


Definition de l’immeuble …



-	Définition du droit fiscal immobilier.
