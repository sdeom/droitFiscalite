
% !TeX root = droitFiscalité.tex

Différence entre taxe et redevance ()
cf. cours du 26/5/2020 44'
124L du 28/6/1981
LOF 1/8/2001

Aujourd'hui 4 catgories de prelevement
Pouvoir legislatif
\begin{enumerate}
  \item impots Gaton Jez (1936) << prelevement pecuniaire requis des particuliers, par voie d'autorité a titre definitif et sans contrepartie dans le but d'assurer la couverture des charges publiques de lEtat des collect et de leur etablissement >>. Une seule exception : la dation en paiement. En 1936 : en 1936 toute personne qui n'est pas l'Etat. A tire definitif pour distinguer l'emprunt forcé (impot << secherese >> et impot << rigueur >> par exemple ne sont pas des impots mais des emprunts). sans contrepartie : c'est un point fondamental.
  \item taxe : pareil que l'impot sauf qu'il y a une contre partie. Plus precisement il y a un service proposé en échange du paiement, qu'on en profite ou non. Elle demeure du par voie d'autorité.
\end{enumerate}

pouvoir réglementaire
\begin{enumerate}
  \item cotisation sociale
  \item rémunération pour service rendu. C'est le nouveau nom pour la redevance (en conjonction de 1958 et 2001)
  Outters 1958. << somme versée par l'usage d'un service public ou d'un ouvrage public et qui trouve sa contrepartie directe et déterminée dans l'utilisation du service public ou de l'ouvrage publique. >>

  Plusieurs dimension interessante : contrepartie directe -> on paye que si on utilise (CE 1970 Augé)
  contrepartie déterminée -> soit le cout du service, soit moins cher ex. : 
\end{enumerate}

INTRODUCTION

-	Présentation générale du système fiscal français et des grands problèmes de finances publiques ;

-	Présentation générale du système fiscal français et des grands problèmes de finances publiques ;


Qui paie ?


Tout le monde. La rév. De 1789 est une rev. Fiscale.


Pourquoi ?


Parcequ’on est obligé

Encadrement des paiements fait en 1992.

3 grandes catégories d’adm

ODAC … central

ODAL local

ODAS sécurité social


ODAC : en France c’est l’Etat et ses établissements publics

ODAL : en France : 13 régions, départements, communes, EPCI, leurs établissements publics

ODAS : 4 branches


Combien ?

Tout est fait pour que l’on ne sache pas : anesthésie fiscale


Adm	Recette	Dépense	Déficit	Dette

ODAC	320 M	509 M	189 M	1 855 M

ODAL	213 M	214 M	1 M	200 M

ODAS	498 M	535 M	37 M	250 M

Total	1 031 M	1 258 M	227 M	2 305 M


Les ODAL ont obligation de voter un budget en équilibre.

En théorie on ne peut emprunter que pour investir, et dans ce cas l’emprunt est neutre.

PIB = 2 283

Tx de pression fiscale = Prlvmnt obligatoir \ PIB ~ 45 %


USA 26 %

Japon 17 %

Allemagne 43 %

Suède 44%

Danemark 45 %


Tx de dépense publique ~ 55 %

USA 37

Allemagne 44 %

Suède 50 %

Danmak 55%


-	Définition du droit fiscal immobilier.


Definition de l’immeuble …



-	Définition du droit fiscal immobilier.
