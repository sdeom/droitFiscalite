Chapitre IV- Taxe annuelle sur les bureaux, les locaux commerciaux, les entrepôts et les aires de stationnement en Ile-de-France

	Section I- Champ d'application

		I- Territorialité 
		II- Locaux imposables 

	Section II- Mécanisme d’imposition 

I- Détermination de la superficie
II- Tarifs
III- Paiement 
IV - Régime fiscal 

A- Locaux figurant à l'actif d'une entreprise
B- Locaux appartenant à des particuliers

V- Particularités résultant de l’extension de la taxe aux aires de stationnement

	A- Champ d’application
	
		1-Surfaces taxales
a Locaux ou aires couvertes ou non couvertes destinés au stationnement des véhicules
b- Locaux ou aires, couvertes ou non couvertes, annexés à des locaux de bureaux, des locaux commerciaux ou des locaux de stockage 

		2 – Surfaces exonérées 

a-	Surfaces de stationnement taxables, exonérées en raison de leur situation géographique
b-	Surfaces de stationnement taxables exonérées en raison de leur propriétaire utilisateur
c-	Surfaces de stationnement de certains établissements d’enseignement
d-	Surfaces de stationnement taxables exonérées en raison de leur superficie

B- Liquidation de la taxe 

Cas pratique n°24 
Cas pratique n°25
Cas pratique n°26
Cas pratique n°27
