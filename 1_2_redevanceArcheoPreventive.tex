% !TeX root = droitFiscalité.tex

\chapter{Redevance d'archéologie préventive}

L'idée est que certaines personnes dont le produit sera à ... INRAP

C'est le secteur même qui a souhaité une redevance archéologique.

28/12/2001 qui a scindé la redevance en deux :
\begin{itemize}
  \item « filière urbanisme »
  \item « filière culture »
\end{itemize}

\section{Champ d'application de la redevance}

  Prévue à L 524-2 cpatri

  \subsection{Redevance « filière urbanisme »}

    \subsubsection{Nature des travaux entrant dans le champ d’application}

      L 524-2 cpatri << sont soumis ... la réalisation de trvx qui affectent le sous-sol >>

      Sont donc exclus un certain nombre de constructions :
      \begin{itemize}
        \item sans fondation : chapiteau, abri de jardin
        \item fondation déjà existante
        \item rénovation
        \item surévélation
        \item changement de destination
      \end{itemize}
      Le seul problème juridique sérieux concerne les places de stationnement, circulaire du 23/6/2005 indique l'exclusion du chmaps d'application. 2018 nvlle circulaire disant le contraire.

    \subsubsection{Exonérations}

      Prévus à L 524-3 cpatri. Cet article ... pour sa part départementale
      \begin{enumerate}
        \item construction destiné à un service public ou d'utilité publique\footnote{établissement de santé ESPIC fevrier 2020} ;
        \item construction d'habitation ou d'hébargement qui béncie d'un PLAI
        \item constructions agricoles
        \item les aménagements prévus par un PPR, tant naturel que technologique
        \item les reconstructions à l'identiques
        \item construction dont la surface est inférieure à moins de 5 m2
      \end{enumerate}
      Il existe une exonération d'usage pour les contructeurs d'habitation pour eux m^me. Ils peuvent faire une demande toujours accordé.

  \subsection{Redevance « filière culture »}

    \subsubsection{Nature des travaux entrant dans le champ d’application}

      L 524-2 cpatri :
      \begin{itemize}
        \item travauxitem
        \item qui affectent le sous-ol
        \item donne lieu a une étude d'impact au sens cenv ou affailloumement qui necess
      \end{itemize}
      10000m2 plus de 50cm de profondeur

    \subsubsection{Exonérations}

      L524-3 cpatri
      \begin{itemize}
        \item trvx agricolesite
        \item trvx forestier
        \item trvsx qui previennent des catastophes
      \end{itemize}

\section{Liquidation de la redevance}

  Deux sujets : le fait générateur et le montant. Il ne faut pas confondre le fait générateur et la liquidation.

  \subsection{Fait générateur}

    \paragraph{Pour la redevance « filière urbanisme»)}
      C'est lacte qui va permettre le début des trvx : autorisation de construire, non opposition \etc

      Quid dans le cadre d'une construction en infraction. Dans ce cas c'est le \PV qui constate l'infraction

    \subsubsection{Travaux soumis à étude d'impact, affouillements autorisés et demandes de diagnostic (redevance « filière culture »)}

      C'est l'arrêté prefectoral d'installation. C'est l'acte qui autorise la réalisation du projet après enquête publique.

  \subsection{Montant de la redevance}

    Si on est exonéré il faut tout de m^me faire la déclaration, dans laquelle on demande l'exonération. Si on entre pas dans le champs d'application il n'y pas de declaratio à faire.

    L 524-7 c par

    \subsubsection{Travaux autorisés en application du Code de l'urbanisme (redevance « filière urbanisme »)}

      C'est de loin le mécanisme le plus compliqué : on détermine une assiette << est constituée par la valeur de l'ensemble immobilier déterminée dans les conditions prévues aux articles L. 331-10 à L. 331-13 du code de l'urbanisme. >> On multiplie une surface qui ressemble à une Carrez à un montant forfaitaire qui dépend de la localisation et de la catégorie de la construction.
      Le toute est multiplié par 0,40

    \subsubsection{Travaux soumis à étude d'impact, déclarations d'affouillements et demandes de diagnostic (Redevance « filière culture »)}

      C'est un montant par m2 (55centimes en 2019)
